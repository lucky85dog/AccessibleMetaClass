\newif\iflatextortf

\iflatextortf
	% tell latex2tortf if this is an article or report
	\documentclass[12pt,letterpaper]{article}
	% set margins
\usepackage[margin=1 in]{geometry}

% use citations
\usepackage[sort]{natbib}

% change the heading of the bibliography
\renewcommand{\bibsection}{\section{References}}

% redefine \pdftooltip so that it behaves differently with and without latextortf
\newcommand{\pdftooltip}[3][]{#2}

% redefine the checkmark
\newcommand{\checkmark}{y\relax}

% redefine the degree symbol to remove dependency on gensymb package
\newcommand{\degree}{\ensuremath{^\circ}}

% redefine booktabs commands
\newcommand{\toprule}{\hline}
\newcommand{\midrule}{\hline}
\newcommand{\bottomrule}{\hline}

% redefine \href
\newcommand{\href}[2]{#1~ (\url{#2})}

% redefine \subfloat to match the \subfigure environment
\usepackage{subfigure}
\makeatletter
\newcommand{\subfloat}[3][]{\subfigure[{#1}]{#2}{#3}}
\makeatother

% redefine \todo so that it gives something useful
\newcommand{\todo}[2][]{\textbf{To Do:} #2}
\else
	\documentclass[10pt,article,twocolumn,tagged]{meta}
\fi

% new commands
\newcommand{\fn}[1]{\emph{#1}}
\newcommand{\packagename}[1]{\textbf{\emph{#1}}}
\newcommand{\envname}[1]{\textbf{\texttt{#1}}}

\author{Andy Clifton}
\title{So you want to replace your WYSIWYG word processor with LaTeX, but 508 compliance has you stumped?}

\begin{document}

\pagenumbering{roman}
\section*{Abstract}
`What you see is what you get' (WYSIWYG) word processors are frequently used within organizations to produce PDF documents that have a common look and feel. The PDFs are tagged, structured, and pass automated tests for document accessibility. PDFs that are created with \LaTeX\ do not pass such tests, making it difficult to use \LaTeX\ in a corporate environment. This document explains how \LaTeX\ can be used to prepare documents that can be used in a corporate environment. This is achieved by using a file source that can be converted in to an \fn{.rtf} file, which can be read in to a WYSIWYG word processor. The same source can be used to create \fn{.pdf} files that satisfy accessibility requirements. This document is intended to be used as a template as it contains most of the elements of a technical \LaTeX\ document, including complex document structures, lists, equations, figures and code listings.

\clearpage
\pagenumbering{arabic}
\tableofcontents
\listoffigures
\listoftables
\clearpage

\section{Introduction}
The de-facto standard for scientific publishing is \LaTeX. \LaTeX\ is often preferred over WYSIWYG word processors for technical documents because of the relatively simple file format that can be shared across users on many different platforms, and the ease of formatting a document for journal publication. In contrast, many organizations have publishing workflows built around WYSIWYG tools, which often include PDF creators. WYSIWYG word processing tools often include the ability to edit a document and track input from co-workers or dedicated editors, which simplifies the review process. Although changes can be compared between \LaTeX documents using command line tools, editors in a corporate environment may lack the training to edit \LaTeX\ source files. So, \LaTeX\ does not fit well in a corporate document preparation workflow unless \LaTeX\ files can be converted into something that can be read by a WYSIWYG editor.

Another issue with using \LaTeX\, is document \emph{accessibility}. Accessibility is important for documents produced by federally-funded organizations: since the US Congress passed the 1998 Section 508 Amendment to the Rehabilitation Act of 1973, it has been a requirement that all federally-funded documents are accessible to people with disabilities. An accessible PDF has several characteristics:

\begin{itemize}
\item All of the document content has been tagged
\item It is possible to define a reading order based on those tags
\item Images and links are given alternate text descriptions
\item Tables are tagged, so that the table structure can be established
\item Unicode descriptions of all characters are required
\end{itemize}

A document that has these characteristics is often referred to as being `508 compliant'. This adds another challenge to the use of \LaTeX\ in a corporate environment: not only does a document need to be converted into something that a WYSIWYG word processor can read, but the \fn{.pdf} output from \LaTeX\ needs to be 508-compliant. As 508-compliance is often judged using automated tests on the \fn{.pdf} file, there is no option to work around this by using careful text descriptions of figures, for example.

In this document, we explain how \LaTeX can be used in a corporate environment where the review process is centered around WYSIWYG tools. To do this, a custom class file is used to generate a range of document classes with common formatting. The \fn{.tex} files can be converted into formats that can be shared with coworkers who use WYSIWYG tools. We explain how the same source file can be used to generate a 508-compliant \fn{*.pdf}, direct from \LaTeX. We also show some of the limitations of the method and what remains to be fixed. Our goal is that this will be a `living' document and template that can be updated as we gain new insight into this process.

\section{Requirements}
The requirements identified in the introduction can be summarized as follows:
\begin{itemize}
\item A single \LaTeX\ class file that ensures that any written document produced using \LaTeX\ conforms to the corporate brand
\item The tex file can be converted into a format that can be read and edited by a WYSIWYG tool
\item The PDF file produced from \LaTeX\ should passes automated accessibility tests in Adobe Acrobat
\end{itemize}

\section{A workflow for producing accessible .pdf files}
The following workflow satisfies the requirements set out in the previous section:
\begin{itemize}
\item Define a class file that contains the correct formatting, etc, using article, report or book classes. Include the minimum number of up-to-date packages in the class and add the \packagename{nag} package to make sure that all users can see that those packages are not deprecated.
\item Provide tools to add accessibility support to the \fn{.pdf} file
\item Create a template showing how to use the class file
\item Create a source code repository for the class and template files, and distribute the URL of the repository to LaTeX users
\item Create documents using the standard class file on local or network drives, or using collaborative tools like writeLaTeX
\item Convert the tex files to rtf using LaTeX2rtf, available at http://latex2rtf.sourceforge.net
\item Get edits and peer reviews done on a cleaned-up .rtf file that has been saved as .doc or .docx
\item Transfer edits from the .doc or .docx document back in to 'tex manually, and complete the PDF production in LaTeX.
\end{itemize}

The rest of this section gives details on each of the above steps.

\subsection{A ``meta'' LaTeX style file}\label{sec:meta.cls}
A class file called \fn{meta.cls} has been written to implement common formatting requirements in \LaTeX\ across the standard \LaTeX\ article, report, book, and memoir classes. The class file calls a common group of packages (Table \ref{Tab:Packages}) regardless of the chosen class. Slightly different options are chosen for those packages if the document uses the memoir class, or if the document uses chapters or not. The class can be customized to meet specific formatting requirements. 

This approach gives a single file which contains all of the formatting required to produce a range of documents. For example, an organization that wants to implement an in-house \LaTeX\ publishing system can customize the \fn{meta.cls} and then make the class file available through a subversion server from which authors pick up the current class file. Each user can then install the class file in their tex structure or in the directory with their files. Similarly, changes are easy to implement in the style file and roll out to users, and because documents are \LaTeX-based it is easy to recompile old documents.

The \packagename{meta} class is written so that it will accept all of the standard global \LaTeX\ options, as well as options that are specific to the underlying class that the user has chosen. For example, to write a draft report, use \verb+\documentclass[draft, report]{meta}+.

Options that have been implemented include:
\begin{description}
\item[book]{compile the document using the \LaTeX\ \emph{book} class. This is intended for longer documents and allows the use of chapters.}
\item[report]{compile the document using the \LaTeX\ \emph{report} class. This is intended for longer documents and allows the use of chapters.}
\item[article]{compile the document using the \LaTeX\ \emph{article} class. This is intended for shorter documents such as journal articles. This class does not support the use of chapters.}
\item[memoir]{compile the document using the \LaTeX\ \emph{memoir} class. This option is not recommended because of the challenge with later converting to rtf format.}
\item[draft]{add a `draft' watermark to all pages and colours all links in blue.}
\item[10pt, 12pt]{set the font size accordingly. The default is 12 point.}
\item[letterpaper, a4paper]{set the paper size. the default is letter paper.}
\end{description}

Table \ref{Tab:Packages} lists the packages that are explicitly called by \packagename{meta} in the order they are called in. These packages often call other packages, so this is not an exhaustive list.

\begin{table*}[htp]
\centering
\caption[Packages explicitly loaded by the meta class]{Packages explicitly loaded by the \emph{meta} class. Unless otherwise stated, packages are not supported by \texttt{latex2rtf}.}
\label{Tab:Packages}
\begin{tabular*}{\textwidth}{p{0.125\textwidth}p{0.1\textwidth}p{0.4\textwidth}p{0.1\textwidth}p{0.2\textwidth}}
\toprule
Packages & options & functionality & \texttt{latex2rtf} support & \texttt{latex2rtf} workaround\\
\midrule
nag & & checks that packages are up to date and looks for bad habits in \LaTeX\ code. & &\\
geometry & & sets page size and margins & \checkmark &\\
mathptmx& & changes fonts & & \\
helvet& & changes fonts & &\\
courier& & changes fonts & &\\
amsfonts, amssymb & & supplies fonts that are useful for mathematics & &\\
booktabs & & & &\\
graphicx & &graphics handling, including \emph{.eps} figures (see Section \ref{sec:Alttext}) & \checkmark &\\
natbib & sort &handles citations and allows the \verb+\cite+, \verb+\citep+ and \verb+\citet+ citation commands. & \checkmark &\\
fontenc & T1 & & &\\
xcolor & & & &\\
babel & english & & &\\
subfigure & & provides the \texttt{subfigure} environment to produce sub figures & \checkmark & \\
hyphenat & & & &\\
setspace & & & &\\
parskip & & & &\\
toclof & subfigure & & & \\
toclifbind & nottoc, notlot, notlof & & &\\
todonotes & & inline and margin to-do notes & \checkmark & Notes are prefaced with \textbf{To Do:} in the output\\
listings & & & &\\
caption & & & &\\
cmap & & & &\\
pdfcomment & & tool-tips. Also calls the package \packagename{hyperref} & \checkmark & the tool tip is suppressed \\
accessibility & & provides tags and document structure & & \\
\bottomrule
\end{tabular*}
\end{table*}

The meta class also includes a few options related to the PDF files that pdflatex produces:
\begin{description}
\item[pdfinterwordspaceon]{Turns on inter-word spacing in PDFs. Requires TexLive 2014.}
\item[pdfminorversion=8]{Sets the PDF version.}
\end{description}

\subsection{Accessibility support}
\LaTeX\ does not prepare a structured PDF document directly. Instead, we use the \packagename{accessibility} package to do this for us. This generates a tagged PDF that passes most automated document tests. The \packagename{accessibility.sty} package can be downloaded from \href{http://www.babs.gmxhome.de/download/da_pdftex/accessibility.sty}{http://www.babs.gmxhome.de/download/da\_pdftex/accessibility.sty}. Note that line 131 in \packagename{accessibility.sty} should be commented out (\fn{pdfoptionpdfminorversion} is set in the \fn{meta.cls}).

\subsubsection{Alternative text}\label{sec:Alttext}
Alternative text, or `Alt text', is a textual description of an equation, link or figure that can be used to replace the visual information in that element. This is often seen as a text `pop-up' in PDF readers. For example, passing the pointer over the following readers should reveal a pop-up:
\begin{equation}
\pdftooltip{a^2+b^2=c^2}{An equation}
\end{equation}

Alt text can be added after the PDF is compiled using a PDF editor such as Adobe's Acrobat Pro. Alternatively -- and probably best for ensuring that the final document is what the author intended -- it can be generated from within the source document using the \envname{pdftooltip} environment from the \packagename{pdfcomment} package. The previous equation was generated using \verb?\pdftooltip{a^2+b^2=c^2}{An equation}?.

The same approach can be used to create alt text for images. For example, Figure \ref{fig:TestimagesWithAltText} has been labeled with a tool tip. 

\begin{figure*}[htp]
\centering
\subfigure[A chick.\label{fig:ChickWithAltText}]{\pdftooltip{\includegraphics[height=2.5in]{Chick1}}{A bright yellow toy model of a chick}}
~
\subfigure[Another chick.\label{fig:ChickWithAltText2}]{\pdftooltip{\includegraphics[height=2.5in]{Chick1}}{A second image of a bright yellow toy model of a chick}}
\caption{Test images}\label{fig:TestimagesWithAltText}
\end{figure*}

The \packagename{accessibility} package includes an \verb+\alt{}+ environment which is intended to create a tool tip. Although it has been included in the source of the next equation, it does not currently work.
\begin{equation}
\alt{a squared plus b squared equals c squared}
a^2+b^2=c^2
\end{equation}

\subsubsection{Problems with embedded fonts}
One requirement of passing automated tests for accessibility is that fonts must be embedded in the the final PDF. You can check the PDF for embedded fonts using a PDF viewer. For example, in Adobe Acrobat Reader, look at the `fonts' tag of the document properties. If any fonts are not shown as being an \emph{embedded subset}, you need to try again. 

Encapsulated postscript figures are particularly prone to having undefined fonts. Check by compiling your document in draft mode, and seeing if the fonts are still present in the output PDF. To fix this problem, you could consider changing the \emph{.eps} file to a \emph{.png}. If you wish to do this `on the fly', you could use this approach in your preamble:

\begin{verbatim}
\usepackage{epstopdf}
\epstopdfDeclareGraphicsRule
 {.eps}{png}{.png}{convert eps:\SourceFile.\SourceExt png:\OutputFile}
\AppendGraphicsExtensions{.png}
\end{verbatim}

\subsection{A template}
The code used to produce this document is available from \href{https://github.com/AndyClifton/AccessibleMetaClass}{https://github.com/AndyClifton/AccessibleMetaClass}. 

\subsection{Changing to a WYSIWYG-readable format using Latex2rtf}
\packagename{latex2rtf} reads \fn{.tex} files and converts common \LaTeX commands into their rich text format equivalent. It can be downloaded from \url{http://latex2rtf.sourceforge.net}.

We want to use the same source file for the PDF and for the conversion to a WYSIWYG-readable format. To ensure compatibility between the \fn{.tex} file and \packagename{latex2rtf}, we limit the packages used in the \packagename{meta} class to those that can be understood by \packagename{latex2rtf}. Also, \packagename{latex2rtf} can read \envname{input} files and has some capacity to deal with \envname{\textbackslash newcommand}s. Therefore, we have created a file to remap all of the environments that are available using the \packagename{meta} class to simpler forms that can be understood by \packagename{latex2rtf}. This file is called \fn{Latex2rtf\_meta\_with\_acc.tex}. 

If you are using the command line version of Latex2rtf, you can convert your latex file using the commands,

\begin{verbatim}
latex foo.tex
latex foo.tex
latex2rtf foo
\end{verbatim}

You may get some error messages, which may or may not be important. Check the file was produced by opening the \fn{.rtf} file in your favorite WYSIWYG package. If it works, save it in that proprietary format, and distribute it for your non-\LaTeX ing coworkers.

The following are suggested as best practices when working with \packagename{latex2rtf}:

\begin{description}
\item[Convert early and often.] Check that your document converts using \packagename{latex2rtf} every time you use  a new environment.
\item[Check new packages.]  If you add a new package to the \fn{meta.cls} file (not recommended), try the conversion immediately, and then again once you actually use one of the new environments or commands. If you've added a new package that \packagename{latex2rtf} doesn't support, you may need to edit the file \fn{Latex2rtf\_meta\_with\_acc.tex} to redefine those commands to something that will get you through the conversion (and review).
\item[Avoid custom commands.] \packagename{latex2rtf} does not work well with custom commands. A list of all recognized commands is available in the manual at \url{http://latex2rtf.sourceforge.net/latex2rtf.pdf}. If you have custom commands, you may need to redefine those to work with the commands that \packagename{latex2rtf} does recognize, and place these in the preamble. To separate your commands from those that are required for compatibility with the \packagename{meta} class, place your commands in a separate file (e.g \fn{Latex2rtf\_meta\_with\_acc\_CUSTOM.tex}).
\item[Remove all of your to-do notes.] We've all forgotten to do this at one time or another...
\end{description}

\section{Unified source document}
So far we have identified two discrete paths that our document will take. These are 
\begin{enumerate}
\item Compiling the \LaTeX\ document using normal \packagename{pdftex}
\item Converting to something portable using \packagename{latex2rtf}. 
\end{enumerate}
And, to keep things tidy, we would like to use the same source for each step. 

Fortunately, \packagename{latex2rtf} offers a boolean, \envname{\textbackslash iflatextortf}. If \packagename{latex2rtf} is used, \envname{\textbackslash iflatextortf} will be TRUE. This means that we can create a preamble that accommodates both paths:

\begin{verbatim}
\newif\iflatextortf

\iflatextortf
    \documentclass[10pt]{report}
    \input{Latex2rtf_meta_with_acc.tex}
\else
    \documentclass[draft, report]{meta} 
\fi
\end{verbatim}

\subsection{Including code listings}
The \packagename{listings} package is one of several packages that can be used to typeset source code, and is used in this document. Unfortunately, \packagename{listings} is not compatible with \packagename{latex2rtf}, and so we need a workaround that will be available to both \packagename{latex2rtf} and \packagename{pdflatex}. 

Using an example from Werner posted on tex.stackexchange.com, we default to the \envname{verbatim} environment, and only use \packagename{listings} if we are compiling with \packagename{pdflatex}:

\begin{verbatim}%
\makeatletter
\@ifundefined{lstlisting}{}{%
  \let\verbatim\relax%
  \lstnewenvironment{verbatim}{
      \lstset{language=[LaTeX]TeX,
      ... % here we can customize the listing
  }}{}%
}
\makeatother
\end{verbatim}

\subsection{The final preamble}

\begin{verbatim}%
\newif\iflatextortf

\iflatextortf
    \documentclass[10pt, letterpaper]{report}
    \input{Latex2rtf_meta_with_acc.tex}
\else
    \documentclass[draft, report]{meta} 
\fi

\makeatletter
\@ifundefined{lstlisting}{}{%
  \let\verbatim\relax%
  \lstnewenvironment{verbatim}{
      \lstset{language=[LaTeX]TeX,
      ... % here we can customize the listing
  }}{}%
}
\makeatother

\author{Andy Clifton}
\title{So you want to replace your WYSIWYG word processor but 508 compliance has you stumped}
 
\begin{document}

\end{verbatim}

\section{Problems with this approach}
There are a few things that still need to be fixed.

\begin{itemize}
\item Problems associated with the \packagename{accessibility} package:
\begin{itemize}
\item Formatting in section headings such as \verb+\emph{}+ breaks compiling
\item Empty \fn{.bbl} files also break it
\item There's an extra line at the start of \envname{lstlisting} environments
\item The \verb+\alt{}+ macro does not work.
\end{itemize}
\end{itemize}

\section{Conclusions}
In order to generate accessible \fn{.pdf} files with \LaTeX\ that can also be shared with WYSIWYG word processor users, a \LaTeX\ meta class was created. This meta class uses a well-defined set of packages that limits the scope of the casual 'tex author to fiddle, whilst giving the author the tools to create complex technical documents. Accessible \fn{.pdf} files are compiled directly from the \LaTeX source code, while \fn{.rtf} files are written using the \packagename{latex2rtf} program. This meets the users' need for a single source file and minimal intervention.

\section*{Acknowledgements}
This document benefitted from contributions to the website, \url{http://tex.stackexchange.com/}.

\end{document}
% bibliography
\cleardoublepage
\bibliographystyle{plainnat}
%\bibintoc
\label{sec:Bib}
\bibliography{bibliography}

\end{document}